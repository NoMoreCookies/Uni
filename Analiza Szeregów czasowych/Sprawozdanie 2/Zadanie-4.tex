% Options for packages loaded elsewhere
\PassOptionsToPackage{unicode}{hyperref}
\PassOptionsToPackage{hyphens}{url}
\documentclass[
  12pt,
]{article}
\usepackage{xcolor}
\usepackage[margin=1in]{geometry}
\usepackage{amsmath,amssymb}
\setcounter{secnumdepth}{5}
\usepackage{iftex}
\ifPDFTeX
  \usepackage[T1]{fontenc}
  \usepackage[utf8]{inputenc}
  \usepackage{textcomp} % provide euro and other symbols
\else % if luatex or xetex
  \usepackage{unicode-math} % this also loads fontspec
  \defaultfontfeatures{Scale=MatchLowercase}
  \defaultfontfeatures[\rmfamily]{Ligatures=TeX,Scale=1}
\fi
\usepackage{lmodern}
\ifPDFTeX\else
  % xetex/luatex font selection
\fi
% Use upquote if available, for straight quotes in verbatim environments
\IfFileExists{upquote.sty}{\usepackage{upquote}}{}
\IfFileExists{microtype.sty}{% use microtype if available
  \usepackage[]{microtype}
  \UseMicrotypeSet[protrusion]{basicmath} % disable protrusion for tt fonts
}{}
\makeatletter
\@ifundefined{KOMAClassName}{% if non-KOMA class
  \IfFileExists{parskip.sty}{%
    \usepackage{parskip}
  }{% else
    \setlength{\parindent}{0pt}
    \setlength{\parskip}{6pt plus 2pt minus 1pt}}
}{% if KOMA class
  \KOMAoptions{parskip=half}}
\makeatother
\usepackage{color}
\usepackage{fancyvrb}
\newcommand{\VerbBar}{|}
\newcommand{\VERB}{\Verb[commandchars=\\\{\}]}
\DefineVerbatimEnvironment{Highlighting}{Verbatim}{commandchars=\\\{\}}
% Add ',fontsize=\small' for more characters per line
\usepackage{framed}
\definecolor{shadecolor}{RGB}{248,248,248}
\newenvironment{Shaded}{\begin{snugshade}}{\end{snugshade}}
\newcommand{\AlertTok}[1]{\textcolor[rgb]{0.94,0.16,0.16}{#1}}
\newcommand{\AnnotationTok}[1]{\textcolor[rgb]{0.56,0.35,0.01}{\textbf{\textit{#1}}}}
\newcommand{\AttributeTok}[1]{\textcolor[rgb]{0.13,0.29,0.53}{#1}}
\newcommand{\BaseNTok}[1]{\textcolor[rgb]{0.00,0.00,0.81}{#1}}
\newcommand{\BuiltInTok}[1]{#1}
\newcommand{\CharTok}[1]{\textcolor[rgb]{0.31,0.60,0.02}{#1}}
\newcommand{\CommentTok}[1]{\textcolor[rgb]{0.56,0.35,0.01}{\textit{#1}}}
\newcommand{\CommentVarTok}[1]{\textcolor[rgb]{0.56,0.35,0.01}{\textbf{\textit{#1}}}}
\newcommand{\ConstantTok}[1]{\textcolor[rgb]{0.56,0.35,0.01}{#1}}
\newcommand{\ControlFlowTok}[1]{\textcolor[rgb]{0.13,0.29,0.53}{\textbf{#1}}}
\newcommand{\DataTypeTok}[1]{\textcolor[rgb]{0.13,0.29,0.53}{#1}}
\newcommand{\DecValTok}[1]{\textcolor[rgb]{0.00,0.00,0.81}{#1}}
\newcommand{\DocumentationTok}[1]{\textcolor[rgb]{0.56,0.35,0.01}{\textbf{\textit{#1}}}}
\newcommand{\ErrorTok}[1]{\textcolor[rgb]{0.64,0.00,0.00}{\textbf{#1}}}
\newcommand{\ExtensionTok}[1]{#1}
\newcommand{\FloatTok}[1]{\textcolor[rgb]{0.00,0.00,0.81}{#1}}
\newcommand{\FunctionTok}[1]{\textcolor[rgb]{0.13,0.29,0.53}{\textbf{#1}}}
\newcommand{\ImportTok}[1]{#1}
\newcommand{\InformationTok}[1]{\textcolor[rgb]{0.56,0.35,0.01}{\textbf{\textit{#1}}}}
\newcommand{\KeywordTok}[1]{\textcolor[rgb]{0.13,0.29,0.53}{\textbf{#1}}}
\newcommand{\NormalTok}[1]{#1}
\newcommand{\OperatorTok}[1]{\textcolor[rgb]{0.81,0.36,0.00}{\textbf{#1}}}
\newcommand{\OtherTok}[1]{\textcolor[rgb]{0.56,0.35,0.01}{#1}}
\newcommand{\PreprocessorTok}[1]{\textcolor[rgb]{0.56,0.35,0.01}{\textit{#1}}}
\newcommand{\RegionMarkerTok}[1]{#1}
\newcommand{\SpecialCharTok}[1]{\textcolor[rgb]{0.81,0.36,0.00}{\textbf{#1}}}
\newcommand{\SpecialStringTok}[1]{\textcolor[rgb]{0.31,0.60,0.02}{#1}}
\newcommand{\StringTok}[1]{\textcolor[rgb]{0.31,0.60,0.02}{#1}}
\newcommand{\VariableTok}[1]{\textcolor[rgb]{0.00,0.00,0.00}{#1}}
\newcommand{\VerbatimStringTok}[1]{\textcolor[rgb]{0.31,0.60,0.02}{#1}}
\newcommand{\WarningTok}[1]{\textcolor[rgb]{0.56,0.35,0.01}{\textbf{\textit{#1}}}}
\usepackage{graphicx}
\makeatletter
\newsavebox\pandoc@box
\newcommand*\pandocbounded[1]{% scales image to fit in text height/width
  \sbox\pandoc@box{#1}%
  \Gscale@div\@tempa{\textheight}{\dimexpr\ht\pandoc@box+\dp\pandoc@box\relax}%
  \Gscale@div\@tempb{\linewidth}{\wd\pandoc@box}%
  \ifdim\@tempb\p@<\@tempa\p@\let\@tempa\@tempb\fi% select the smaller of both
  \ifdim\@tempa\p@<\p@\scalebox{\@tempa}{\usebox\pandoc@box}%
  \else\usebox{\pandoc@box}%
  \fi%
}
% Set default figure placement to htbp
\def\fps@figure{htbp}
\makeatother
\setlength{\emergencystretch}{3em} % prevent overfull lines
\providecommand{\tightlist}{%
  \setlength{\itemsep}{0pt}\setlength{\parskip}{0pt}}
\usepackage[OT4]{polski}
\usepackage[utf8]{inputenc}
\usepackage{graphicx}
\usepackage{float}
\usepackage{amsthm}
\newtheorem{definition}{Definicja}[section]
\usepackage{bookmark}
\IfFileExists{xurl.sty}{\usepackage{xurl}}{} % add URL line breaks if available
\urlstyle{same}
\hypersetup{
  pdftitle={Sprawozdanie 2},
  pdfauthor={Kacper Szmigielski (282255)},
  hidelinks,
  pdfcreator={LaTeX via pandoc}}

\title{Sprawozdanie 2}
\usepackage{etoolbox}
\makeatletter
\providecommand{\subtitle}[1]{% add subtitle to \maketitle
  \apptocmd{\@title}{\par {\large #1 \par}}{}{}
}
\makeatother
\subtitle{Analiza Szeregów Czasowych}
\author{Kacper Szmigielski (282255)}
\date{}

\begin{document}
\maketitle

{
\setcounter{tocdepth}{2}
\tableofcontents
}
\listoffigures
\listoftables
\section{Zadanie 4}\label{zadanie-4}

\begin{center}\includegraphics{Zadanie-4_files/figure-latex/przekształcenia_1-1} \end{center}

Wykres autoplot nie sugeruje znacznych wzrostów wariancji szeregu
czasowego wraz z upływem czasu.

Sugerowana \textbf{lambda} dla transformacji boxa-coxa wynosi 2

Zastosujemy tranformacje z tym parametrem żeby sprawdzić, czy ma ona
istotny wpływ na dane

\begin{figure}[H]

{\centering \includegraphics{Zadanie-4_files/figure-latex/po_trans_boxa_coxa-1} 

}

\caption{Dane po tranformacji boxa-coxa}\label{fig:po_trans_boxa_coxa}
\end{figure}

Dane \ref{fig:po_trans_boxa_coxa} po tranformacji Boxa-Coxa wyglądają
tak samo, co potwierdza jej zbędność. Dalszą analizę będziemy
przeprowadzać na danych, bez tranformacji boxa-coxa.

Teraz sprawdzamy sugerowane wartości różnicowań

Funkcja ndiffs proponuje 1 różnicowań z opóźnieniem 1.

Natomiast funkcja \textbf{nsdiffs} proponuje 1 różnicowań z opóźnieniem
sezonowym 4

Wykonujemy proponowane różnicowania na danych

\begin{Shaded}
\begin{Highlighting}[]
\NormalTok{euretail.learn}\FloatTok{.4} \OtherTok{\textless{}{-}} \FunctionTok{diff}\NormalTok{(euretail.learn, }\AttributeTok{lag=}\DecValTok{4}\NormalTok{)}
\NormalTok{euretail.learn.}\FloatTok{4.1} \OtherTok{\textless{}{-}} \FunctionTok{diff}\NormalTok{(euretail.learn}\FloatTok{.4}\NormalTok{, }\AttributeTok{lag=}\DecValTok{1}\NormalTok{)}
\end{Highlighting}
\end{Shaded}

Po zróżnicowaniu nasz szereg ma postać:

\begin{figure}[H]

{\centering \includegraphics{Zadanie-4_files/figure-latex/po_zróżnicowani-1} 

}

\caption{Wykres danych euretail po zastosowaniu transformacji}\label{fig:po_zróżnicowani}
\end{figure}

Szereg \ref{fig:po_zróżnicowani} po zastosowanych tranformacjach ma
charakter białoszumowy. Tylko jedna odstająca obserwacja funkcji ACF,
brak regularności.

\section{- - - - - - - - - - - - - - - - - - - - - - - - - - - - - - - -
- - - - - - -}\label{section}

\section{Identyfikacja modeli AR(p) oraz MA(q) oraz
ARIMA}\label{identyfikacja-modeli-arp-oraz-maq-oraz-arima}

\section{- - - - - - - - - - - - - - - - - - - - - - - - - - - - - - - -
- - - - - - -}\label{section-1}

\section{wykresy ACF, PACF danych uczących po
transformacjach}\label{wykresy-acf-pacf-danych-uczux105cych-po-transformacjach}

\section{za pomocą ggtsdisplay}\label{za-pomocux105-ggtsdisplay}

ggtsdisplay(euretail.learn.4.1.1)

\subsubsection{Czy średnia jest równa 0? (--\textgreater{} ważne dla
uwzględnienia dryfu w modelu
ARIMA)}\label{czy-ux15brednia-jest-ruxf3wna-0-waux17cne-dla-uwzglux119dnienia-dryfu-w-modelu-arima}

mean(euretail.learn.4.1.1) \# 0.0072

\subsubsection{modele na bazie PACF oraz ACF to odpowiednio AR(4) oraz
MA(9)}\label{modele-na-bazie-pacf-oraz-acf-to-odpowiednio-ar4-oraz-ma9}

\subsubsection{oznaczane jako}\label{oznaczane-jako}

\subsubsection{{[}model.ar{]} oraz
{[}model.ma{]}}\label{model.ar-oraz-model.ma}

model.ar \textless- Arima(euretail.learn, order=c(4,2,0),
seasonal=c(0,1,0))

model.ma \textless- Arima(euretail.learn, order=c(0,2,9),
seasonal=c(0,1,0))

\#\_\_\_\_\_\_\_\_\_\_\_\_\_\_\_\_\_\_\_\_\_\_\_\_\_\_\_\_\_\_\_\_\_\_\_\_\_\_\_\_\_\_\_\_\_\_\_\_\_\_\_\_\_\_\_\_\_\_\_\_\_\_\_\_
\# UWAGA 1: \# W przypadku modeli ARIMA ewentualnie można zastanowiać
się też \# nad AR(1), MA(5), MA(4) albo MA(1) \# (ale nie umiszczać ich
w sprawozdaniu, jeśli nie wnoszą \# nic ciekawego)
\#\_\_\_\_\_\_\_\_\_\_\_\_\_\_\_\_\_\_\_\_\_\_\_\_\_\_\_\_\_\_\_\_\_\_\_\_\_\_\_\_\_\_\_\_\_\_\_\_\_\_\_\_\_\_\_\_\_\_\_\_\_\_\_\_

\#\_\_\_\_\_\_\_\_\_\_\_\_\_\_\_\_\_\_\_\_\_\_\_\_\_\_\_\_\_\_\_\_\_\_\_\_\_\_\_\_\_\_\_\_\_\_\_\_\_\_\_\_\_\_\_\_\_\_\_\_\_\_\_\_\_
\# UWAGA 2: \# Skoro średnia = 0.0072 \neq 0, to warto sprawdzić też
modele \# z dryfem dodając argument ``include.drift = TRUE'' w funkcji
Arima \# Zobacz Przykład 3: Dopasowanie modelu ARIMA \# dla szeregu gnp
(R skrypt)
\#\_\_\_\_\_\_\_\_\_\_\_\_\_\_\_\_\_\_\_\_\_\_\_\_\_\_\_\_\_\_\_\_\_\_\_\_\_\_\_\_\_\_\_\_\_\_\_\_\_\_\_\_\_\_\_\_\_\_\_\_\_\_\_\_\_

\section{model automatyczny ARIMA}\label{model-automatyczny-arima}

\section{oznaczane jako}\label{oznaczane-jako-1}

\section{{[}model.auto{]} - ustawiamy dodatkowo stepwise=FALSE,
approximation=FALSE}\label{model.auto---ustawiamy-dodatkowo-stepwisefalse-approximationfalse}

\section{aby dokładniej przeszukować potencjalne modele, zgodnie z
uwagą}\label{aby-dokux142adniej-przeszukowaux107-potencjalne-modele-zgodnie-z-uwagux105}

\section{w dokumentacji funkcji
auto.arima():}\label{w-dokumentacji-funkcji-auto.arima}

\section{}\label{section-2}

\section{''}\label{section-3}

\section{If you are analysing just one time series, and can afford to
take
some}\label{if-you-are-analysing-just-one-time-series-and-can-afford-to-take-some}

\section{more time, it is recommended that you
set}\label{more-time-it-is-recommended-that-you-set}

\section{stepwise=FALSE and
approximation=FALSE.}\label{stepwisefalse-and-approximationfalse.}

\section{''}\label{section-4}

\section{}\label{section-5}

model.auto \textless- auto.arima(euretail.learn, stepwise=FALSE,
approximation=FALSE) \# ARIMA(0,1,3)(0,1,1){[}4{]}

\#\_\_\_\_\_\_\_\_\_\_\_\_\_\_\_\_\_\_\_\_\_\_\_\_\_\_\_\_\_\_\_\_\_\_\_\_\_\_\_\_\_\_\_\_\_\_\_\_\_\_\_\_\_\_\_\_\_\_\_\_\_\_\_\_\_\_\_\_\_\_\_
\# UWAGA 3: \# Warto sprawdzić modele automatyczne optymalizujące
``aicc'' albo ``bic''. \# Zobacz Przykład 2: dopasowanie modelu ARIMA z
uwzględnieniem oceny \# istotności współczynników (R skrypt)
\#\_\_\_\_\_\_\_\_\_\_\_\_\_\_\_\_\_\_\_\_\_\_\_\_\_\_\_\_\_\_\_\_\_\_\_\_\_\_\_\_\_\_\_\_\_\_\_\_\_\_\_\_\_\_\_\_\_\_\_\_\_\_\_\_\_\_\_\_\_\_\_

\#\_\_\_\_\_\_\_\_\_\_\_\_\_\_\_\_\_\_\_\_\_\_\_\_\_\_\_\_\_\_\_\_\_\_\_\_\_\_\_\_\_\_\_\_\_\_\_\_\_\_\_\_\_\_\_\_\_\_\_\_\_\_\_\_\_\_\_\_\_\_\_\_\_\_\_
\# UWAGA 4: \# Warto sprawdzić istotność wspołczynników otrzymanych
modeli. \# Czy zostawienie tylko współczynników istotnych poprawia
jakość dopasowania \# modelu? \# Zobacz Przykład 2: dopasowanie modelu
ARIMA z uwzględnieniem oceny \# istotności współczynników (R skrypt)
\#\_\_\_\_\_\_\_\_\_\_\_\_\_\_\_\_\_\_\_\_\_\_\_\_\_\_\_\_\_\_\_\_\_\_\_\_\_\_\_\_\_\_\_\_\_\_\_\_\_\_\_\_\_\_\_\_\_\_\_\_\_\_\_\_\_\_\_\_\_\_\_\_\_\_\_

\section{- - - - - - - - - - - - - - - - - - - - - - - - - - - - - - - -
- - - - - - -}\label{section-6}

\section{Identyfikacja modeli dekompozycji
tslm()}\label{identyfikacja-modeli-dekompozycji-tslm}

\section{- - - - - - - - - - - - - - - - - - - - - - - - - - - - - - - -
- - - - - - -}\label{section-7}

\section{{[}model.tslm.0{]} oraz {[}model.tslm.1{]} oraz
{[}model.tslm.2{]}}\label{model.tslm.0-oraz-model.tslm.1-oraz-model.tslm.2}

model.tslm.0 \textless- tslm(euretail.learn \textasciitilde{} trend)
model.tslm.1 \textless- tslm(euretail.learn \textasciitilde{} season +
trend) model.tslm.2 \textless- tslm(euretail.learn \textasciitilde{}
season + trend + I(trend\^{}2))

\section{- - - - - - - - - - - - - - - - - - - - - - - - - - - - - - - -
- - - - - - -}\label{section-8}

\section{Identyfikacja modeli algorytmów wygładzania wykładniczego
(modele
ETS)}\label{identyfikacja-modeli-algorytmuxf3w-wygux142adzania-wykux142adniczego-modele-ets}

\section{- - - - - - - - - - - - - - - - - - - - - - - - - - - - - - - -
- - - - - - -}\label{section-9}

\section{proste wygładzanie wykładnicze (algorytm SES) oznaczane jako
{[}model.ses{]}}\label{proste-wygux142adzanie-wykux142adnicze-algorytm-ses-oznaczane-jako-model.ses}

model.ses \textless- ses(euretail.learn, h=h, initial=``simple'')

\section{modele z algorytmu Holta oznaczane jako {[}model.holt{]} oraz
{[}model.holt.damped{]}}\label{modele-z-algorytmu-holta-oznaczane-jako-model.holt-oraz-model.holt.damped}

model.holt \textless- holt(euretail.learn, h=h) model.holt.damped
\textless- holt(euretail.learn, h=h, damped=TRUE)

\section{modele z algorytmu Holta-Wintersa oznaczane jako
{[}model.hw.add{]} oraz
{[}model.hw.mult{]}}\label{modele-z-algorytmu-holta-wintersa-oznaczane-jako-model.hw.add-oraz-model.hw.mult}

model.hw.add \textless- hw(euretail.learn, seasonal=``additive'')
model.hw.mult \textless- hw(euretail.learn, seasonal=``multiplicative'')

\section{model ETS (automatyczny za pomocą ets()) oznaczany jako
{[}model.ets{]}}\label{model-ets-automatyczny-za-pomocux105-ets-oznaczany-jako-model.ets}

model.ets \textless- ets(euretail.learn) model.ets1 \textless-
ets(euretail.learn, model=``MAM'') \# niekoniecznie sensowny wybór

\#\_\_\_\_\_\_\_\_\_\_\_\_\_\_\_\_\_\_\_\_\_\_\_\_\_\_\_\_\_\_\_\_\_\_\_\_\_\_\_\_\_\_\_\_\_\_\_\_\_\_\_\_\_\_\_\_\_\_\_\_\_\_\_\_\_\_\_\_\_\_\_\_\_\_\_
\# UWAGA 5: \# Możliwe, że wystarczy w sprawozdaniu rozważyć tylko
{[}model.tslm.2{]} \# przy tym wyborze zbioru treningowego, bo na nim
trend \# nie jest tam opisywany przez funkcję nieliniową
\#\_\_\_\_\_\_\_\_\_\_\_\_\_\_\_\_\_\_\_\_\_\_\_\_\_\_\_\_\_\_\_\_\_\_\_\_\_\_\_\_\_\_\_\_\_\_\_\_\_\_\_\_\_\_\_\_\_\_\_\_\_\_\_\_\_\_\_\_\_\_\_\_\_\_\_

\#\_\_\_\_\_\_\_\_\_\_\_\_\_\_\_\_\_\_\_\_\_\_\_\_\_\_\_\_\_\_\_\_\_\_\_\_\_\_\_\_\_\_\_\_\_\_\_\_\_\_\_\_\_\_\_\_\_\_\_\_\_\_\_\_\_\_\_\_\_\_\_\_\_\_\_
\# UWAGA 6: \# Zainteresowani mogą sprawdzić też inne modele ses tj. z
\# np. initial=``optimal'' lub innymi zmienionymi parametrami, \# ale w
samym sprawozdaniu wystarczy zasadniczo jeden ses \# (ten, który uważamy
za najlepszy)
\#\_\_\_\_\_\_\_\_\_\_\_\_\_\_\_\_\_\_\_\_\_\_\_\_\_\_\_\_\_\_\_\_\_\_\_\_\_\_\_\_\_\_\_\_\_\_\_\_\_\_\_\_\_\_\_\_\_\_\_\_\_\_\_\_\_\_\_\_\_\_\_\_\_\_\_

\#\_\_\_\_\_\_\_\_\_\_\_\_\_\_\_\_\_\_\_\_\_\_\_\_\_\_\_\_\_\_\_\_\_\_\_\_\_\_\_\_\_\_\_\_\_\_\_\_\_\_\_\_\_\_\_\_\_\_\_\_\_\_\_\_\_\_\_\_\_\_\_\_\_\_\_
\# UWAGA 7: \# wystarczy model.ets, ale jeśli jesteśmy w stanie podać
lepszy model ets \# np. ustawiając odpowiednio parametry, to można to
pokazać w sprawozdaniu \# ale nie jest to jednak konieczne.
\#\_\_\_\_\_\_\_\_\_\_\_\_\_\_\_\_\_\_\_\_\_\_\_\_\_\_\_\_\_\_\_\_\_\_\_\_\_\_\_\_\_\_\_\_\_\_\_\_\_\_\_\_\_\_\_\_\_\_\_\_\_\_\_\_\_\_\_\_\_\_\_\_\_\_\_

\#\_\_\_\_\_\_\_\_\_\_\_\_\_\_\_\_\_\_\_\_\_\_\_\_\_\_\_\_\_\_\_\_\_\_\_\_\_\_\_\_\_\_\_\_\_\_\_\_\_\_\_\_\_\_\_\_\_\_\_\_\_\_\_\_\_\_\_\_\_\_\_\_\_\_\_\_\_\_\_
\# \#--------------- 3 - Wyznaczenie prognoz dla zbioru testowego
------------------
\#\_\_\_\_\_\_\_\_\_\_\_\_\_\_\_\_\_\_\_\_\_\_\_\_\_\_\_\_\_\_\_\_\_\_\_\_\_\_\_\_\_\_\_\_\_\_\_\_\_\_\_\_\_\_\_\_\_\_\_\_\_\_\_\_\_\_\_\_\_\_\_\_\_\_\_\_\_\_\_
\# Wyznaczenie prognoz dla zbioru testowego na podstawie dopasowanych
modeli \# oraz wybranej metody referencyjnej (tzn. wybranej naiwnej
metody \# prognozowania). ( \# \# Wskazówka: można zastosować funkcje
meanf(), naive(), snaive() lub rwf() \# z pakietu forecast. \# - - - - -
- - - - - - - - - - - - - - - - - - - - - - - - - - - - - - - - - -

\section{}\label{section-10}

\section{1. prognozy dla modeli arima}\label{prognozy-dla-modeli-arima}

\section{1.1 ar}\label{ar}

\section{1.2 ma}\label{ma}

\section{1.3 auto}\label{auto}

\section{2. prognozy dla modeli
dekompozycji}\label{prognozy-dla-modeli-dekompozycji}

\section{2.1 tslm0}\label{tslm0}

\section{2.2 tslm1}\label{tslm1}

\section{3. prognozy dla modeli ets}\label{prognozy-dla-modeli-ets}

\section{3.1 ses}\label{ses}

\section{3.2 holt + holt damped}\label{holt-holt-damped}

\section{3.3 hw-add + hw-mult}\label{hw-add-hw-mult}

\section{3.4 ets}\label{ets}

\section{4 metody referencyjne}\label{metody-referencyjne}

\section{4.1 prognoza oparta na
średniej}\label{prognoza-oparta-na-ux15bredniej}

\section{4.2 standardowa metoda naiwna}\label{standardowa-metoda-naiwna}

\section{4.3 sezonowa metoda naiwna}\label{sezonowa-metoda-naiwna}

\section{4.4 metoda uwzględniająca
dryf}\label{metoda-uwzglux119dniajux105ca-dryf}

\section{}\label{section-11}

\section{Wskazówka:}\label{wskazuxf3wka}

\section{Zobacz plik}\label{zobacz-plik}

\section{Przyklad1\_prognozowanie\_szeregow\_czasowych.R}\label{przyklad1_prognozowanie_szeregow_czasowych.r}

\section{do listy 2}\label{do-listy-2}

\section{1.1}\label{section-12}

forecast.model.ar \textless- forecast::forecast(model.ar, h=h) \# 1.2
forecast.model.ma \textless- forecast::forecast(model.ma, h=h)

\section{1.3}\label{section-13}

forecast.model.auto \textless- forecast::forecast(model.auto, h=h)

\section{2.1}\label{section-14}

forecast.model.tslm0 \textless- forecast::forecast(model.tslm.0, h=h) \#
2.2 forecast.model.tslm1 \textless- forecast::forecast(model.tslm.1,
h=h) \# 2.3 forecast.model.tslm2 \textless-
forecast::forecast(model.tslm.2, h=h)

\section{3.1}\label{section-15}

forecast.model.ses \textless- forecast::forecast(model.ses, h=h) \# 3.2
forecast.model.holt \textless- forecast::forecast(model.holt, h=h)
forecast.model.holt.damped \textless-
forecast::forecast(model.holt.damped, h=h) \# 3.3 forecast.model.hw.add
\textless- forecast::forecast(model.hw.add, h=h) forecast.model.hw.mult
\textless- forecast::forecast(model.hw.mult, h=h) \# 3.4
forecast.model.ets \textless- forecast::forecast(model.ets, h=h)
forecast.model.ets.1 \textless- forecast::forecast(model.ets1, h=h)

\section{4.1}\label{section-16}

forecast.mean \textless- meanf(x=euretail.learn, h=8) \# 4.2
forecast.naive \textless- naive(x=euretail.learn, h=8) \# 4.3
forecast.snaive \textless- snaive(x=euretail.learn, h=8) \# 4.4
forecast.rwf \textless- rwf(x=euretail.learn, h=8, drift=TRUE)

\#\_\_\_\_\_\_\_\_\_\_\_\_\_\_\_\_\_\_\_\_\_\_\_\_\_\_\_\_\_\_\_\_\_\_\_\_\_\_\_\_\_\_\_\_\_\_\_\_\_\_\_\_\_\_\_\_\_\_\_\_\_\_\_\_\_\_\_\_\_\_\_\_\_\_\_\_\_\_\_
\# \#------------- 4 - Przedstawienie skonstruowanych prognoz na
wykresie-----------
\#\_\_\_\_\_\_\_\_\_\_\_\_\_\_\_\_\_\_\_\_\_\_\_\_\_\_\_\_\_\_\_\_\_\_\_\_\_\_\_\_\_\_\_\_\_\_\_\_\_\_\_\_\_\_\_\_\_\_\_\_\_\_\_\_\_\_\_\_\_\_\_\_\_\_\_\_\_\_\_
\# Przedstawienie skonstruowanych prognoz na wykresie \# i porównanie z
wartościami rzeczywistymi (zbiór testowy). \# - - - - - - - - - - - - -
- - - - - - - - - - - - - - - - - - - - - - - - - -

\section{1.1}\label{section-17}

autoplot(forecast.model.ar) + autolayer(euretail.test) \# 1.2
autoplot(forecast.model.ma) + autolayer(euretail.test) \# 1.3
autoplot(forecast.model.auto) + autolayer(euretail.test)

\section{2.1}\label{section-18}

autoplot(forecast.model.tslm0) + autolayer(euretail.test) \# 2.2
autoplot(forecast.model.tslm1) + autolayer(euretail.test) \# 2.3
autoplot(forecast.model.tslm2) + autolayer(euretail.test)

\section{3.1}\label{section-19}

autoplot(forecast.model.ses) + autolayer(euretail.test) \# 3.2
autoplot(forecast.model.holt) + autolayer(euretail.test)
autoplot(forecast.model.holt.damped) + autolayer(euretail.test) \# 3.3
autoplot(forecast.model.hw.add) + autolayer(euretail.test)
autoplot(forecast.model.hw.mult) + autolayer(euretail.test) \# 3.4
autoplot(forecast.model.ets) + autolayer(euretail.test)
autoplot(forecast.model.ets.1) + autolayer(euretail.test)

\section{4.1}\label{section-20}

autoplot(forecast.mean) + autolayer(euretail.test) \# 4.2
autoplot(forecast.naive) + autolayer(euretail.test) \# 4.3
autoplot(forecast.snaive) + autolayer(euretail.test) \# 4.4
autoplot(forecast.rwf) + autolayer(euretail.test)

\#\_\_\_\_\_\_\_\_\_\_\_\_\_\_\_\_\_\_\_\_\_\_\_\_\_\_\_\_\_\_\_\_\_\_\_\_\_\_\_\_\_\_\_\_\_\_\_\_\_\_\_\_\_\_\_\_\_\_\_\_\_\_\_\_\_\_\_\_\_\_\_\_\_\_\_\_\_\_\_
\# \#---------------------- 5 - Porównanie dokładności prognoz
--------------
\#\_\_\_\_\_\_\_\_\_\_\_\_\_\_\_\_\_\_\_\_\_\_\_\_\_\_\_\_\_\_\_\_\_\_\_\_\_\_\_\_\_\_\_\_\_\_\_\_\_\_\_\_\_\_\_\_\_\_\_\_\_\_\_\_\_\_\_\_\_\_\_\_\_\_\_\_\_\_\_
\# Porównanie dokładności prognoz dla zbioru testowego i uczącego \# z
uwzględnieniem wybranych miar oceny dokładności, np. RMSE, MAE, MAPE i
MASE. \# \# Wskazówka: można wykorzystać funkcję accuracy() z pakietu
forecast. \# - - - - - - - - - - - - - - - - - - - - - - - - - - - - - -
- - - - - - - - -\#

\section{1.1}\label{section-21}

accuracy(forecast.model.ar, euretail.test) \# 1.2
accuracy(forecast.model.ma, euretail.test) \# 1.3
accuracy(forecast.model.auto, euretail.test)

\section{2.1}\label{section-22}

accuracy(forecast.model.tslm0, euretail.test) \# 2.2
accuracy(forecast.model.tslm1, euretail.test) \# 2.3
accuracy(forecast.model.tslm2, euretail.test)

\section{3.1}\label{section-23}

accuracy(forecast.model.ses, euretail.test) \# 3.2
accuracy(forecast.model.holt, euretail.test)
accuracy(forecast.model.holt.damped, euretail.test) \# 3.3
accuracy(forecast.model.hw.add, euretail.test)
accuracy(forecast.model.hw.mult, euretail.test) \# 3.4
accuracy(forecast.model.ets, euretail.test)
accuracy(forecast.model.ets.1, euretail.test)

\section{4.1}\label{section-24}

accuracy(forecast.mean, euretail.test) \# 4.2 accuracy(forecast.naive,
euretail.test) \# 4.3 accuracy(forecast.snaive, euretail.test) \# 4.4
accuracy(forecast.rwf, euretail.test)

\#\_\_\_\_\_\_\_\_\_\_\_\_\_\_\_\_\_\_\_\_\_\_\_\_\_\_\_\_\_\_\_\_\_\_\_\_\_\_\_\_\_\_\_\_\_\_\_\_\_\_\_\_\_\_\_\_\_\_\_\_\_\_\_\_\_\_\_\_\_\_\_\_\_\_\_\_\_\_\_
\# \#----------- 6 - Wnioski dotyczące wyboru optymalnego modelu
------------------
\#\_\_\_\_\_\_\_\_\_\_\_\_\_\_\_\_\_\_\_\_\_\_\_\_\_\_\_\_\_\_\_\_\_\_\_\_\_\_\_\_\_\_\_\_\_\_\_\_\_\_\_\_\_\_\_\_\_\_\_\_\_\_\_\_\_\_\_\_\_\_\_\_\_\_\_\_\_\_\_
\# Wnioski dotyczące wyboru optymalnego modelu. \# Które podejście do
modelowania wydaje się bardziej adekwatne \# dla rozważanego szeregu
czasowego? \# - - - - - - - - - - - - - - - - - - - - - - - - - - - - -
- - - - - - - - - -\#

\section{Do samodzielnego wykonania.}\label{do-samodzielnego-wykonania.}

\end{document}
